\documentclass{resume}

\newcommand{\en}[1]{#1}
\newcommand{\zh}[1]{}

\zh{\usepackage{xeCJK}}
\zh{\setCJKmainfont{思源宋体}}
\zh{\setCJKsansfont{思源黑体}}
\zh{\setCJKmonofont{思源黑体}}

\begin{document}

\name{\en{Haizhi Geng}\zh{耿海直}}
\basicInfo{
      \email{ghzpotato@gmail.com} \textperiodcentered\
      \phone{+(86) 155-9376-9871} \textperiodcentered\
      \github[JmPotato]{https://github.com/JmPotato} \textperiodcentered\
      \homepage[Blog]{https://ipotato.me}
}

\section{\faGraduationCap\ \en{Education}\zh{教育经历}}
\en{\datedsubsection{\textbf{Beijing University of Posts and Telecommunications}, Undergraduate}{09/2017 -- 06/2021}}
\zh{\datedsubsection{\textbf{北京邮电大学}, 本科}{2017/09 -- 2021/06}}
\begin{itemize}
      \item \en{Major: Computer Science and Technology, School of Computer. Graduation date: 06/2021}
            \zh{计算机科学与技术,计算机学院,2021 年 6 月毕业}
\end{itemize}

\section{\faUsers\ \en{Work Experience}\zh{工作经历}}
\en{\datedsubsection{\textbf{\href{https://www.bytedance.com/en/}{ByteDance Inc.}}, Beijing, China}{09/2019 -- 04/2020}}
\zh{\datedsubsection{\textbf{\href{https://www.bytedance.com/zh/}{北京字节跳动科技有限公司(ByteDance Inc.)}}}{2019/09 -- 2020/04}}
\en{\role{App Back-end Research \& Develop}{R\&D Intern, Golang/Python}}
\zh{\role{App 后端研发}{后端研发实习}}
\begin{itemize}
      \item \en{Responsible for App R\&D work related to vertical business, using Golang/Python language, Thrift and other tools for microservice development and maintenance.}
            \zh{负责懂车帝 App 垂直业务相关的后端研发工作,使用 Go/Python 语言,Thrift 等工具进行微服务的开发与维护}
      \item \en{Participated in R\&D work of UGC and live broadcast, had cross-departmental cooperative development experience with Toutiao, Douyin and Watermelon Video.}
            \zh{参与社区 UGC 和直播相关业务研发,有与头条,抖音和西瓜视频等团队进行过跨部门合作开发经验}
      \item \en{Learned a lot about Web development and DevOps.}
            \zh{学到了很多关于 Web 开发与 DevOps 的知识}
\end{itemize}

\en{\datedsubsection{\textbf{\href{https://pingcap.com/en/}{PingCAP Inc.}}, Beijing, China}{06/2020 -- Present}}
\zh{\datedsubsection{\textbf{\href{https://pingcap.com/zh/}{北京平凯星辰科技发展有限公司(PingCAP Inc.)}}}{2020/06 -- 至今}}
\en{\role{Database R\&D}{Scheduling R\&D Intern}}
\zh{\role{数据库研发}{研发工程师实习}}
\begin{itemize}
      \item \en{Responsible for the development of TiDB's scheduling component PD. Tuned and refactored some modules, designed and participated in the development and maintenance of distributed TSO services across data centers.}
            \zh{负责 TiDB 调度组件 PD 的研发工作。调优重构了部分模块,设计并参与了跨数据中心分布式授时服务的开发与维护}
      \item \en{Responsible for the development of TiDB. Participated in the design, development and testing of the function/syntax for the Stale Read, i.e, non-consistent read.}
            \zh{负责 TiDB 的研发工作。参与了 Stale Read 非一致性读的功能语法设计、开发与测试等工作}
      \item \en{Learned a lot about the distributed system and database.}
            \zh{学到了很多关于分布式系统和数据库的知识}
\end{itemize}
\en{\role{Database R\&D}{Scheduling R\&D}}
\zh{\role{数据库研发}{研发工程师}}
\begin{itemize}
      \item \en{Responsible for the design, development and testing of TSO Scalability related features. Optimized and alleviated the performance bottleneck of TSO service caused by Go Runtime scheduling in high concurrency scenarios, improved TiDB transaction performance by 260\%+ TPS, and reduced latency by 80\% milliseconds.}
            \zh{负责 TSO Scalability 相关功能的设计,开发和测试。优化缓解了高并发场景下 Go Runtime 调度导致的 TSO 服务性能瓶颈问题,提升 TiDB 事务性能 260\%+ TPS,降低延迟 80\% 毫秒}
      \item \en{Refactored the Storage module of the PD component to standardize the underlying abstraction and decouple multiple modules, enhancing the pluggability of the underlying storage and the upper layer application.}
            \zh{重构了 PD 组件的 Storage 模块,规范底层抽象并解耦多个模块,加强了底层存储和上层应用的可插拔性}
\end{itemize}

\section{\faGithubAlt\ \en{Portfolios}\zh{个人项目}}
\datedsubsection{\textbf{MVCC 时光机}}{\url{https://github.com/Long-Live-the-DoDo/rfc}}
\en{Flashback function based on the TiDB MVCC}
\zh{基于 TiDB MVCC 特性实现的 Flashback 功能}
\begin{itemize}
      \item \en{Won third prize at TiDB Hackathon 2021.}
            \zh{项目获得 TiDB Hackathon 2021 三等奖}
      \item \en{Based on TiDB MVCC data to enable sub-second data archival, with the ability to restore specified tables to any existing historical version via Flashback SQL.}
            \zh{基于 TiDB 的 MVCC 数据实现了亚秒级别的数据回档,能够通过 Flashback SQL 将指定表还原至现存的任意历史版本}
\end{itemize}

\datedsubsection{\textbf{wince-tp(W.I.P)}}{\url{https://github.com/JmPotato/wince-tp}}
\en{A thread pool implementation that supports Rust asynchronous programming}
\zh{一个支持 Rust 异步编程的线程池实现}
\begin{itemize}
      \item \en{At present, only a prototype is implemented based on the work-stealing scheduling algorithm, after then I want to transform it into a TPC-like sharding model.}
            \zh{目前只基于工作窃取调度算法实现了一个雏形,后续想改造成类似 TPC 的 Sharding 模型}
\end{itemize}

\datedsubsection{\textbf{fp-growth-rs}}{\url{https://github.com/JmPotato/fp-growth-rs}}
\en{An implementation of the FP-Growth algorithm in pure Rust, which is inspired by enaeseth/python-fp-growth.}
\zh{FP-Growth 算法的 Rust 实现,项目灵感来自于 enaeseth/python-fp-growth}
\begin{itemize}
      \item \en{A Rust implementation for frequent pattern mining algorithm, completed for the purpose of graduation design, and then found that Rust lacks similar libraries, so make it open-source.}
            \zh{一个用于进行频繁项集挖掘算法的 Rust 实现,出于实现毕业设计的目的完成,后发现 Rust 缺少同类库,遂将其单独开源}
\end{itemize}

\datedsubsection{\textbf{dopamine}}{\url{https://github.com/JmPotato/dopamine}}
\en{Python Web Framework}
\zh{Python Web 框架}
\begin{itemize}
      \item \en{Based on WSGI of PEP 3333.}
            \zh{基于 PEP 3333 的 WSGI 实现}
      \item \en{Use the gevent library to provide fast network IO performance.}
            \zh{使用 gevent 库提供快速的网络 IO 表现}
      \item \en{Use Flask-like router and HTTP wrapper.}
            \zh{使用了类 Flask 的路由功能和 HTTP 封装}
\end{itemize}

\section{\faCogs\ \en{Skills}\zh{技能}}
\begin{itemize}[parsep=0.25ex]
      \item \en{\textbf{Programming Language}:
                  \textbf{multilingual} (not limited to any specific language), 
                  experienced in Golang/Rust/Python, 
                  comfortable with JavaScript/C/C++}
            \zh{\textbf{编程语言}:
                  \textbf{泛语言}(编程不受特定语言限制),
                  熟悉 Golang/Rust/Python,
                  了解 JavaScript/C/C++ 等}

      \item \en{\textbf{Distributed System/Database}:
                  Experience in tuning and deployment of TiDB, know the basic use of K8s and tidb-operator.
                  taken course MIT 6.824 and PingCAP's Talent Plan,
                  understand the basic theory of distributed system/database,
                  including but not limited to algorithms such as Raft and Percolator.}
            \zh{\textbf{分布式系统/数据库}:
                  有分布式数据库 TiDB 的调优开发以及部署经验,了解 K8s 以及 tidb-operator 的基本使用。
                  自主学习了 MIT 6.824 和 PingCAP's Talent Plan 等课程,
                  了解分布式系统/数据库的基本理论,包括但不限于 Raft 和 Percolator 等算法}

      \item \en{\textbf{Developing Tool}:
                  familiar with Linux-based programming,
                  have experience with team tools like Git, Jira, etc.}
            \zh{\textbf{开发工具}:
                  熟悉 Linux,有 Git、Jira 等团队协作工具的使用经验}

      \item \en{\textbf{Others}:
                  have experience using MySQL/Redis/Kafka, 
                  understand Docker and Docker orchestration concepts, 
                  have experience in developing digital currency quantitative trading strategy, 
                  solved JobShop problem with genetic algorithm.}
            \zh{\textbf{其它}:
                  有 MySQL/Redis/Kafka 使用经验,了解容器及容器编排相关概念,
                  开发过数字货币量化交易策略,使用遗传算法解决过 JobShop 问题}
\end{itemize}

\section{\faInfo\ \en{Miscellaneous}\zh{杂项}}
\begin{itemize}[parsep=0.25ex]
      \item \en{\textbf{Open-source Contributions}: contributed to \texttt{@rust-analyzer @etcd @tikv @tidb @talent-plan}, etc.}
            \zh{\textbf{开源贡献}: 为 \texttt{@rust @rust-analyzer @etcd @tikv @tidb @talent-plan} 等知名开源项目组织贡献过代码}
      \item \en{\textbf{Passionate about sharing}: I have been writing on my personal blog for years and have accumulated several high-quality technical sharing articles.}
            \zh{\textbf{热爱分享}: 在个人博客上常年坚持写作,积累高质量技术分享文章数篇}
      \item \en{\textbf{Language Level}: English CET-6, ability to conduct daily conversation and essay reading, experience in English presentation.}
            \zh{\textbf{语言水平}:英语 CET-6,能够进行日常对话和论文阅读,有英文演讲经验}
      \item \en{\textbf{Personal tags}: self-driven, quick learner, earnest curiosity, open source lover.}
            \zh{\textbf{个人标签}:自驱动,学习能力强,做事认真,保持好奇,热爱开源}
      \item \en{\textbf{Interests}: Distributed System, Database, Cloud and Web application.}
            \zh{\textbf{兴趣领域}:分布式系统、数据库、云以及 Web 应用等}
      \item \en{\textbf{Selected Courses}: OS, Network, Database, Algorithm, Compiler Principle.}
            \zh{\textbf{主修课程}:操作系统、计算机网络、数据库系统原理、算法设计与分析、编译原理}
\end{itemize}

\end{document}
