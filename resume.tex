\documentclass{resume}

\newcommand{\en}[1]{#1}
\newcommand{\zh}[1]{}

\zh{\usepackage{xeCJK}}
\zh{\setCJKmainfont{思源宋体}}
\zh{\setCJKsansfont{思源黑体}}
\zh{\setCJKmonofont{思源黑体}}

\begin{document}

\name{\en{Haizhi Geng}\zh{耿海直}}
\basicInfo{
      \email{ghzpotato@gmail.com} \textperiodcentered\
      \phone{+(86) 155-9376-9871} \textperiodcentered\
      \github[JmPotato]{https://github.com/JmPotato} \textperiodcentered\
      \homepage[Blog]{https://ipotato.me}
}

\section{\faGraduationCap\ \en{Education}\zh{教育经历}}
\en{\datedsubsection{\textbf{Beijing University of Posts and Telecommunications}, Undergraduate}{09/2017 -- Present}}
\zh{\datedsubsection{\textbf{北京邮电大学}, 在读本科}{2017/09 -- 至今}}
\begin{itemize}
      \item \en{Major: Computer Science and Technology, School of Computer. Anticipated graduation date: 06/2021}
            \zh{计算机科学与技术,计算机学院,2021 年毕业}
\end{itemize}

\section{\faUsers\ \en{Work Experience}\zh{工作经历}}
\en{\datedsubsection{\textbf{\href{https://www.bytedance.com/en/}{ByteDance Inc.}}, Beijing, China}{09/2019 -- 04/2020}}
\zh{\datedsubsection{\textbf{\href{https://www.bytedance.com/zh/}{北京字节跳动科技有限公司(ByteDance Inc.)}}}{2019/09 -- 2020/04}}
\en{\role{App Back-end Research \& Develop}{R\&D Intern, Golang/Python}}
\zh{\role{App 后端研发}{后端研发实习}}
\begin{itemize}
      \item \en{Responsible for App R\&D work related to vertical business, using Golang/Python language, Thrift and other tools for microservice development and maintenance.}
            \zh{负责懂车帝 App 垂直业务相关的后端研发工作,使用 Go/Python 语言,Thrift 等工具进行微服务的开发与维护}
      \item \en{Participated in R\&D work of UGC and live broadcast, had cross-departmental cooperative development experience with Toutiao, Douyin and Watermelon Video.}
            \zh{参与社区 UGC 和直播相关业务研发,有与头条,抖音和西瓜视频等团队进行过跨部门合作开发经验}
      \item \en{Learned a lot about Web development and DevOps.}
            \zh{学到了很多关于 Web 开发与 DevOps 的相关知识}
\end{itemize}

\en{\datedsubsection{\textbf{\href{https://aws.amazon.com/}{AWS Chine Inc.}}, Beijing, China}{05/2020 -- 06/2020}}
\zh{\datedsubsection{\textbf{\href{https://www.amazonaws.cn/}{亚马逊 AWS 中国(AWS China Inc.)}}}{2020/05 -- 2020/06}}
\en{\role{Solution Architecture}{Solution Architect Intern}}
\zh{\role{解决方案架构}{解决方案架构师实习}}
\begin{itemize}
      \item \en{Responsible for the communication with business customers, build AWS architecture meets their needs, and provide them with mature cloud architecture solutions.}
            \zh{负责 ToB 客户的技术接洽沟通,搭建符合其需求的 AWS 产品架构,为客户提供成熟的云架构解决方案}
\end{itemize}

\en{\datedsubsection{\textbf{\href{https://pingcap.com/en/}{PingCAP Inc.}}, Beijing, China}{06/2020 -- Present}}
\zh{\datedsubsection{\textbf{\href{https://pingcap.com/zh/}{北京平凯星辰科技发展有限公司(PingCAP Inc.)}}}{2020/06 -- 至今}}
\en{\role{Database R\&D}{Scheduling R\&D Intern}}
\zh{\role{数据库研发}{调度研发实习}}
\begin{itemize}
      \item \en{Responsible for the development of TiDB's scheduling component PD.}
            \zh{负责 TiDB 调度组件 PD 的研发工作}
      \item \en{Learned a lot about Distributed System.}
            \zh{学到了很多关于分布式系统的相关知识}
\end{itemize}

\section{\faGithubAlt\ \en{Portfolios}\zh{个人项目}}
\datedsubsection{\textbf{Infomatter}}{}
\en{Information aggregation system based on machine learning}
\zh{基于机器学习的信息聚合系统}
\begin{itemize}
      \item \en{Won the first prize of the BUPT Innovation and Entrepreneurship Competition, certified as the city-level innovation.}
            \zh{项目荣获北京邮电大学创新创业大赛一等奖,通过北京市级高等院校大创项目认定}
      \item \en{Developed and maintained RESTful API service.}
            \zh{开发维护相关 RESTful API 服务}
      \item \en{Implemented the front-end and back-end services for Infomatter Web.}
            \zh{实现了 Infomatter Web 版的前后端服务}
      \item \en{Main language and framework used are Node.js + Express.}
            \zh{语言和框架为 Node.js + Express}
\end{itemize}

\datedsubsection{\textbf{raft-kv(W.I.P)}}{\url{https://github.com/JmPotato/raft-kv}}
\en{Distributed key-value database based on Raft and Percolator models}
\zh{基于 Raft 与 Percolator 模型的分布式键值数据库}
\begin{itemize}
      \item \en{Learned from MIT 6.824 and PingCAP's Talent Plan.}
            \zh{学习 MIT 6.824 和 PingCAP Talent Plan 后完成}
      \item \en{Implemented Raft distributed consensus algorithm and Google Percolator transaction model in Rust.}
            \zh{使用 Rust 开发,基于 Raft 分布式共识算法和 Google Percolator 事务模型实现}
\end{itemize}

\datedsubsection{\textbf{Pomash}}{\url{https://github.com/JmPotato/Pomash}}
\en{Lightweight blog system based on Tornado Web Framework}
\zh{基于 Tornado Web Framework 的轻量级博客系统}
\begin{itemize}
      \item \en{Based on Tornado Web Framework, use Sqlite3 to store data, separated front-end and back-end models, have certain scalability.}
            \zh{基于 Tornado Web Framework 开发,使用 Sqlite3 存储数据,前后端分离,有一定扩展性}
      \item \en{Support Markdown syntax, LaTeX formula rendering, themes changing, Dropbox backup and other functions.}
            \zh{支持Markdown语法,LaTeX公式渲染,主题更换以及Dropbox备份等功能}
\end{itemize}

\datedsubsection{\textbf{dopamine}}{\url{https://github.com/JmPotato/dopamine}}
\en{Python Web Framework}
\zh{Python Web 框架}
\begin{itemize}
      \item \en{Based on WSGI of PEP 3333.}
            \zh{基于 PEP 3333 的 WSGI 实现}
      \item \en{Use the gevent library to provide fast network IO performance.}
            \zh{使用 gevent 库提供快速的网络 IO 表现}
      \item \en{Use Flask-like router and HTTP wrapper.}
            \zh{使用了类 Flask 的路由功能和 HTTP 封装}
\end{itemize}

% \section{\faLinkedin\ \en{Personal Experience}\zh{个人经历}}
% \en{\datedsubsection{\textbf{BUPT BYRIO Open Source Community}, Beijing}{2018 -- Present}}
% \zh{\datedsubsection{\textbf{北京邮电大学 BYRIO 开源社区}}{2018 -- 至今}}
% \en{\role{Developer}{Tec Department}}
% \zh{\role{开发者}{技术部}}
% \begin{itemize}
%       \item \en{Participate in the development and maintenance of community products, such as: campus gym appointment system and GPA calculation script.}
%             \zh{参加北邮校园相关产品的开发与维护,如:校园健身房预约系统,北邮教务绩点查询计算脚本}
% \end{itemize}

\section{\faCogs\ \en{Skills}\zh{技能}}
\begin{itemize}[parsep=0.25ex]
      \item \en{\textbf{Programming Language}:
                  \textbf{multilingual} (not limited to any specific language), 
                  experienced in Golang/Python, 
                  comfortable with Rust/JavaScript/C/C++}
            \zh{\textbf{编程语言}:
                  \textbf{泛语言}(编程不受特定语言限制),
                  熟悉 Golang/Python,
                  了解 Rust/JavaScript/C/C++ 等}

      \item \en{\textbf{Developing Tool}:
                  familiar with Linux-based programming,
                  have experience with team tools like Trello, Git, etc.}
            \zh{\textbf{开发工具}:
                  熟悉 Linux,有 Trello、Slack, Git 等团队合作工具的使用经验}

      \item \en{\textbf{Distributed System}:
                  taken course MIT 6.824 and PingCAP's Talent Plan,
                  understand consensus algorithms like Raft.}
            \zh{\textbf{分布式系统}:
                  自主学习了 MIT 6.824 和 PingCAP's Talent Plan 课程,
                  了解 Raft 算法}

      \item \en{\textbf{Others}:
                  have experience using MySQL/Redis/Kafka, 
                  understand Docker and Docker orchestration concepts, 
                  have experience in developing digital currency quantitative trading strategy, 
                  solved JobShop problem with genetic algorithm.}
            \zh{\textbf{其它}:
                  有 MySQL/Redis/Kafka 使用经验,了解容器及容器编排相关概念,
                  开发过数字货币量化交易策略,使用遗传算法解决过 JobShop 问题}
\end{itemize}

\section{\faInfo\ \en{Miscellaneous}\zh{杂项}}
\begin{itemize}[parsep=0.25ex]
      \item \en{Personal tags: self-driven, quick learner, earnest curiosity.}
            \zh{个人标签:自驱动,学习能力强,做事认真,保持好奇}
      \item \en{Interests: Distributed System, Cloud, Database and Web application.}
            \zh{兴趣领域:分布式系统、云、数据库、Web 应用等}
      \item \en{Open-source Contributions: contributed to \texttt{@rust-analyzer @tikv @tidb @talent-plan @etcd}, etc.}
            \zh{开源贡献: 为 \texttt{@rust-analyzer @tikv @tidb @talent-plan @etcd} 等项目组织贡献过代码}
      \item \en{Selected Courses: OS, Network, Database, Algorithm, Compiler Principle.}
            \zh{主修课程:操作系统、计算机网络、数据库系统原理、算法设计与分析、编译原理}
      \item \en{Language Level: English CET-6.}
            \zh{语言水平:英语 CET-6}
\end{itemize}

\end{document}
